%
%   Chapter Experiment
%
%   Qing-Cheng Li (r01922024 at csie dot ntu dot edu dot tw)
%   R.O.C.103.07
%
\chapter{實驗結果與分析}
\label{c:exp}

本章節進行了以樣式偵測特性的實驗,
以Wikipedia的文章作為模擬內容串流文件、
以YAGO的資料作為答案、
以PATTY提供的樣式對Wikipedia的文章進行偵測。
並介紹評估的標準以及對實驗結果的分析。

\section{測試資料集}
\label{s:dataset}



% Wikipedia
% YAGO

% PATTY

\section{評估標準}
\label{s:eval}
於本研究希望了解對每一個實體特性,
利用樣式自文章中偵測該特性的效能,
以精確率(Precision)與招回率(Recall)評估使用樣式偵測每個

\section{實驗結果}
\label{s:result}

%\subsecion{}
%錯誤分析
