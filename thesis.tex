\documentclass{ntuthesis}

\usepackage{times}
\usepackage{verbatim}
\usepackage{color}
\usepackage{url}
\usepackage{graphicx}
\usepackage{array}

% Format the refs
\usepackage[sort,comma]{natbib}

\usepackage{titlesec}
\usepackage{titletoc}
\usepackage{etoolbox}

% Centering the ToC's title
\usepackage{tocloft}

% Include ToC/LoF/LoT into ToC
\usepackage[notbib]{tocbibind}

% Include outside .pdf
\usepackage{pdfpages}

% Add wallpaper
\usepackage{wallpaper}

% 2 words indent in first line for Chinese
\usepackage{indentfirst}
\setlength{\parindent}{2em}

% Using the tex-text mapping for ligatures etc.
\defaultfontfeatures{Mapping=tex-text}

% Set the default fonts
\setmainfont{Times New Roman}
%\setCJKmainfont{楷體-繁}
\setCJKmainfont{標楷體}
% value > 0
\def\xeCJKembold{0.4}

% hack into xeCJK, you don't need to understand it
\def\saveCJKnode{\dimen255\lastkern}
\def\restoreCJKnode{\kern-\dimen255\kern\dimen255}

% save old definition of \CJKsymbol and \CJKpunctsymbol for CJK output
\let\CJKoldsymbol\CJKsymbol
\let\CJKoldpunctsymbol\CJKpunctsymbol

% apply pdf literal fake bold
\def\CJKfakeboldsymbol#1{%
  \special{pdf:literal direct 2 Tr \xeCJKembold\space w}%
  \CJKoldsymbol{#1}%
  \saveCJKnode
  \special{pdf:literal direct 0 Tr}%
  \restoreCJKnode}
\def\CJKfakeboldpunctsymbol#1{%
  \special{pdf:literal direct 2 Tr \xeCJKembold\space w}%
  \CJKoldpunctsymbol{#1}%
  \saveCJKnode
  \special{pdf:literal direct 0 Tr}%
  \restoreCJKnode}
\newcommand\CJKfakebold[1]{%
  \let\CJKsymbol\CJKfakeboldsymbol
  \let\CJKpunctsymbol\CJKfakeboldpunctsymbol
  #1%
  \let\CJKsymbol\CJKoldsymbol
  \let\CJKpunctsymbol\CJKoldpunctsymbol}
\newcommand\zhbf[1]{\CJKfakebold{#1}}

% Very Naive Chinese Number
\newcommand\naiveZhNum[1]{
\ifnum #1 = 1 
一 
\else \ifnum #1 = 2
二
\else \ifnum #1 = 3
三
\else \ifnum #1 = 4
四
\else \ifnum #1 = 5
五
\else \ifnum #1 = 6
六
\else \ifnum #1 = 7
七
\else \ifnum #1 = 8
八
\else \ifnum #1 = 9
九
\else
#1
\fi\fi\fi\fi\fi\fi\fi\fi\fi
}


% ToC, LoF, LoT centering settings with package tocloft
\renewcommand{\cftloftitlefont}{\hfill \Huge}
\renewcommand{\cftafterloftitle}{\hfill}
\renewcommand{\cfttoctitlefont}{\hfil \Huge}
\renewcommand{\cftaftertoctitle}{\hfill}
\renewcommand{\cftlottitlefont}{\hfill \Huge}
\renewcommand{\cftafterlottitle}{\hfill}

\titleformat{\chapter}{\centering\Huge\bfseries}{第\naiveZhNum{\thechapter}章}{1em}{}
\renewcommand{\cftchapleader}{\cftdotfill{\cftdotsep}} % dots for chapters
\titlecontents{chapter}[0em]{}{\makebox[4.1em][l]{第\naiveZhNum{\thecontentslabel}章}}{}{\cftdotfill{\cftdotsep}\contentspage}

\makeatletter
\patchcmd{\@chapter}{\addtocontents{lot}{\protect\addvspace{10\p@}}}{}{}{}
\patchcmd{\@chapter}{\addtocontents{lof}{\protect\addvspace{10\p@}}}{}{}{}
\makeatother

% Your information goes here
% author: Tz-Huan Huang [http://www.csie.ntu.edu.tw/~tzhuan]

% ----------------------------------------------------------------------------
% "THE CHOCOLATE-WARE LICENSE":
% Tz-Huan Huang wrote this file. As long as you retain this notice you
% can do whatever you want with this stuff. If we meet some day, and you think
% this stuff is worth it, you can buy me a chocolate in return Tz-Huan Huang
% ----------------------------------------------------------------------------

% modify by Qing-Cheng Li (qcl) [http://qcl.github.io/]

% Syntax: \var{English}{Chinese}
\university{National Taiwan University}{國立臺灣大學}
\collage{College of Electrical Engineering and Computer Science}{電機資訊學院}
\institute{Department of Computer Science and Information Engineering}{資訊工程學系}
\title{Detection of Entity Properties in Content Stream}{內容串流中實體特性偵測之研究}
\author{Qing-Cheng Li}{李卿澄}
\studentid{R01922024}
\advisor{Hsin-Hsi Chen, Ph.D.}{陳信希\ 博士}
\year{2014}{103}
\month{July}{7}
\day{31}


% Modify some default titles for Chinese
\renewcommand{\contentsname}{目錄}
\renewcommand{\listfigurename}{圖目錄}
\renewcommand{\listtablename}{表目錄}
\renewcommand{\tablename}{表}
\renewcommand{\figurename}{圖}
\renewcommand{\bibname}{參考文獻}

\begin{document}

% 臺大論文浮水印
%\CenterWallPaper{0.174}{pdfs/watermark.pdf}
%\setlength{\wpXoffset}{6.1725cm}
%\setlength{\wpYoffset}{10.5225cm}

\frontmatter

\makecover
%\makespine % TODO

% 口試委員會審定書 TODO 
% 以 \makecertification 產生
%\makecertification
% 或匯入外部.pdf檔
%\addcontentsline{toc}{chapter}{口試委員會審定書}
%\includepdf[pages={1}]{pdfs/cert.pdf}

% 誌謝 TODO
%\begin{acknowledgementszh}

\end{acknowledgementszh}

%\begin{acknowledgementsen}
%I'm glad to thank\ldots 
%\end{acknowledgementsen}

% 摘要
\begin{abstractzh}
測試一下摘要

知識產生日新月異

讓知識庫更新很重要

本研究利用以樣式為基礎的

偵測網路串流

是否包含實體的特性

以期可供知識庫更新加速之用

設計了方法,流程等等

\end{abstractzh}

\begin{abstracten}
\end{abstracten}

%\begin{comment}
%\category{I2.10}{Computing Methodologies}{Artificial Intelligence --
%Vision and Scene Understanding} \category{H5.3}{Information
%Systems}{Information Interfaces and Presentation (HCI) -- Web-based
%Interaction.}
%
%\terms{Design, Human factors, Performance.}
%
%\keywords{Region of interest, Visual attention model, Web-based
%games, Benchmarks.}
%\end{comment}


% TODO FIXME - Chinese Chapter Wording setting
% Table of Content
\clearpage
\tableofcontents
% List of Figures
\clearpage
\renewcommand{\numberline}[1]{\figurename~#1\hspace*{1em}}
\listoffigures
% List of Tables
\clearpage
\renewcommand{\numberline}[1]{\tablename~#1\hspace*{1em}}
\listoftables

\mainmatter

% Your thesis goes here
%
%   Chapter Introduction
%
%   Qing-Cheng Li (r01922024 at csie dot ntu dot edu dot tw)
%   R.O.C.103.07
%
\chapter{緒論}
\label{c:intro}

近年來,隨著網際網路的快速發展,網際網路中開始出現越來越多彙整人類知識的網站與資源,
如Wikipedia\footnote{http://www.wikipedia.org/},目前已經有超過280種語言,其中光是英語的條目就超過4,500,000條,
是透過來自世界各地的志願編輯者一字一句的貢獻建立而成的。

除了讓世界各地的人們可以在網際網路上共享知識之外,也讓計算機得以利用人類的知識,
輔助、改善、甚至自動化人工智慧、資料探勘與擷取、知識汲取、自動問答系統等任務。

為了達成此一目的,讓計算機可以看懂人類的知識,於是,
便出現了各式各樣透過擷取人工建立的知識資源,產生的結構化知識資料庫,並彼此相互鏈結。
如圖\ref{i:lod}所示,目前已有不少的結構化知識庫。
這些各式各樣的資源的最源頭,還是透過人類一字一句編輯產生的。% FIXME

\begin{figure}
\centering
\includegraphics[width=0.5\textwidth]{images/01-lod-datasets}
\caption{各種結構化知識資料庫之間的鏈結}
\label{i:lod}
資料來源:linkeddata.org(2011)
\end{figure}

%
%   Background
%
\section{背景介紹}
在現今這個資訊社會,每當有新的事件發生時,例如某人的出生或死亡、某政治人物贏得了選舉、某地發生了天災等等,
這些資訊便會在網路上透過網路新聞、網誌、論壇、社群網站、微網誌等管道流傳。
新事件的發生代表了一些舊有的知識可能需要更新,例如修改某個人物的生死狀態、某個政治人物的勝選記錄等,
如有Wikipedia的志願編輯者注意到這些資訊之後,便會據此更新Wikipedia上關於某人或某球隊的條目。

如Wikipedia這種用來記錄實體(Entity)、實體的特性(Properties)、實體與實體間的關係(Relationships)的資料庫被稱為知識庫(Knowledge Base)。
以Wikipedia來說,Wikipedia以文章的形式儲存了人物、組織、公司、城市、事件等實體,
並在文章的文句中以超連結(Hyperlink)的形式描述實體與實體間的關係。
而文章中的資訊框(Infoboxes)則以半結構化的形式描述了實體的特性,如圖\ref{i:wiki}。

實體間的關係也可以看為是一種實體的特性,
同樣如圖\ref{i:wiki}中的句子「... who was the co-founder, chairman, and CEO of \emph{Apple Inc}.」,
說明了Steven Jobs與Apple公司之間的關係:Jobs是Apple公司的CEO;
同時也可以視為Jobs這個實體具有「CEO of」這樣的特性。

\begin{figure}
\centering
\includegraphics[width=0.65\textwidth]{images/01-wiki-as-kb}
\caption{Wikipedia中的文章與資訊框}
\label{i:wiki}
\end{figure}

除了Wikipedia之外,還有其他DBpedia\citep{dbpedia}、YAGO\citep{yago}、Freebase\citep{freebase}等規模大小不一、不同應用的知識庫。
其中YAGO、DBpedia是透過自動化的方法擷取Wikipedia的內容,Freebase仰賴社群更新,這些知識庫都直接或間接仰賴人力更新。  %FIXME

%
%   Motivation
%
\section{研究動機}
知識庫仰賴志願編輯者的維護與更新,但這些人數比起資料庫中記錄的實體顯然非常的少。
這意味著知識庫的更新總是落後於新知識(過去從來不知道的知識與舊有知識內容的更動)的產生,
當新知識產生一段時日之後,知識庫的維護者才會更新知識庫,
圖\ref{i:wikicitenews} \citep{kba2012}表現了這種落後可能達一年甚至更長。
因此,如何讓人工編輯的速度可以跟上新知識出現的速度便成為一個重要的課題。

\begin{figure}
    \centering
    \includegraphics[width=0.5\textwidth]{images/01-wiki-cite-delay}
    \caption{Time lag for a sample of ~60,000 web pages cited by Wikipedia articles}
    \label{i:wikicitenews}
\end{figure}

為了縮短這個差距,自2012年起至今的每年美國NIST\footnote{National Institue of Standards and Technology}的TREC\footnote{Text Retriveal Conference}都會舉辦知識庫加速(Knowledge Base Acceleration, KBA)競賽,
希望參賽者可以建立一個系統,從由新聞、部落格、論壇等內容所組成、依時間所排列的龐大內容串流\footnote{2012年包含了4,973個小時,2013年包含了11,948小時。\cite{kba2013}每小時包含約100,000份文件。}(Content Strean)中,
對有興趣的實體\footnote{2012年有27個人物、2個組織,2013年有98個人物,12個組織與24個設施。},包含了人物、組織或建築,進行過濾,
推薦文件給知識庫的編輯者,建議編輯者這份文件可能含有新的知識可供更新知識庫。

要從可能包含新知識的內容串流中,過濾出可能可以協助更新知識庫的文件,
則需要知道文章中是否包含知識。知識庫中實體間的關係、實體的特性就是一種知識,
我們想要知道一份來自內容串流的文件是否提及實體的特性,
例如句子「Jobs was born in San Francisco, California on February 24, 1955」中就包含了實體Jobs及其出生地這項特性的知識。
如果可以快速地辨識文章中是否包含這些資訊,推薦給知識庫的編輯者參考,便能夠進一步地加速知識庫的更新與維護。

%
%   Goal
%
\section{研究目標}
本研究的目標是建立一個過濾系統,可以快速地處理內容串流。
對於內容串流內的每一份文件,偵測該文件中是否存在我們感興趣的實體特性。

透過這個系統,我們將知道經過過濾的文件到底是完全沒有提及我們感興趣的實體特性,
或提及其中一種實體特性,甚至是同時提及了多種實體特性。
這些資訊將有助於加速知識庫的更新與維護。

%
%   Structure
%
\section{論文架構}
本論文共分成五個章節。
第一章是緒論,簡介本研究的背景、動機與目標。
第二章是文獻探討,列舉了一些相關的研究與資源。
第三章是研究方法,提出如何於內容串流中究偵測實體特性的方法與步驟。
第四章是實驗結果與分析。    % TODO
第五章是結論與未來展望。    % TODO


%
%   Chapter Related Works
%
%   Qing-Cheng Li (r01922024 at csie dot ntu dot edu dot tw)
%   R.O.C.103.07
%
\chapter{相關研究與文獻}
\label{c:related}

本章將分別介紹知識庫相關的資源與研究,
涵蓋了結構化知識庫(Structural Knowledge Base)、
知識庫加速(Knowledge Base Accerleration)與知識庫的應用(Application of Knowledge Base),
以及樣式與實體間關係的相關研究。

\section{知識庫}
\subsection{結構化知識庫}
相對於Wikipedia以文章來儲存知識,\cite{dbpedia} 的DBpedia、
\cite{yago} 的YAGO以及\cite{freebase} 的Freebase等都以結構化的方式儲存知識。
DBpedia、YAGO與Freebase都提供以TTL(Terse RDF\footnote{Resource Description Framework} Triple Language)格式儲存的資料,
以供處理與利用。

DBpedia是一個大型、多語言的知識庫,自Wikipedia擷取知識。
透過Wikipedia條目內的資訊框(Inforbox),擷取實體的特性,並連結實體。
並進一步整理了資訊框內的特性,以人工將語意相同的實體特性合併。

YAGO除了自Wikipedia擷取資訊外,還搭配了WordNet\footnote{http://wordnet.princeton.edu/},
建立一個輕量、高品質與覆蓋度的實體與關係資料庫。
其以<實體,關係,實體>的形式儲存事實(Facts),即實體與實體間的關係,
並定義關係的領域(Domain)與範圍(Range)、
階層式架構的實體類型(Types)與其他資源如DBpedia、WordNet間的連結。

Freebase是一個用以結構化人類知識的社群協作的可擴展元組(Scalable tuple)資料庫。
與Wikipedia一樣,是由志願者創建與維護其中的資料,並與其它知識庫亦互有連結。

\subsection{知識庫加速}

\cite{kba2012,kba2013}縱覽了2012、2013年的TREC知識庫加速競賽,
連續兩年都有的CCR(Cumulative Citation Recommendation)任務是過濾內容串流,
推薦對更新目標實體有幫助的文章,以縮短知識產生與知識庫更新之間的時間差。
圖\ref{i:kba-corps}是這兩年CCR任務中內容串流的資料量與目標實體的數量,
內容串流中每小時約有100,000份文件。
CCR任務主要專注在尋找文章中是否有出現目標實體(Mentions/Zero Mentions),
再進一步依照有幫助的程度分成四個等級(Garbage/Neutral/Useful/Vital)。

\begin{figure}
    \centering
    \includegraphics[width=0.65\textwidth]{images/02-kba-corpus}
    \caption{TREC知識庫加速競賽資料量}
    \label{i:kba-corps}
\end{figure}

\cite{kba-hltoce}是2012年效能最好的團隊,
以具名實體、單字詞(Unigram)作為特徵訓練SVM分類器,
對每個目標實體訓練一個二元分類器。
把尋找實體的問題作為主題分類(Topic Classification)問題來處理。

\cite{kba-msra}則是2013年效能最好的團隊。
以查詢擴展(Query Expansion)、分類與學習式排序法(Learning to rank)。
其中分類是使用以文件特徵、實體特徵、文件與實體特徵、
時間特徵與引用特徵訓練隨機森林(Random Forest)分類器。

此外,\cite{kba-entity-detection}也專注於尋找串流中的實體,
提出了當出現新的實體時不需要新的訓練資料集的方法。
使用了文件中心特徵、實體資料特徵與時間特徵訓練隨機森林分類器,
以尋找串流中與實體高度相關的文件。
並且以TREC KBA 2012的資料做測試,
效能比\cite{kba-hltoce}還要好。

\subsection{知識庫的應用}

知識庫的應用常見於問答系統或實體消歧義上。

在問答系統的應用上,知識庫提供了一個知識的集合,也就是答案的來源,
\cite{yago-qa}將問句轉換為查詢語言,對YAGO進行查詢並回答問題。
\cite{freebase-qa-extract}利用Freebase的資料,以邏輯形式(Latent Logical Forms)對映問題與答案。
\cite{freebase-qa-parse}將知識庫視為一個實體為點,關聯為邊的圖,
假設答案只會存在於周遭的節點上,成為一個主題圖(Topic Graph),
將問句剖析後在圖上走訪尋找答案。

而在具名實體消歧義(Named Entity Disambiguation)上的應用,
\cite{dbpedia-spotlight} 提供了一個系統將文本內的實體連結至DBpedia。
他們將DBpedia的實體建立向量空間模型(Vector Space Model),
但與傳統VSM不同的是,IDF(Inverse Document Frequency)反應一個字在整個資料集內的重要程度,
卻沒辦法反應一個字在歧義候選選項內的重要程度。
為了要能夠找出一個字對消歧義的能力,
他們提出ICF(Inverse Candidate Frequency)來取代IDF計算字在向量空間中的權重,
定義$ICF(word) = log \frac{|R|}{n(word)}$,其中$|R|$是候選實體的數量,
$n(word)$是字在候選實體中出現的次數。
將DBpedia的實體以向量空間模型建模,以$TF*ICF$計算字的權重,
將文句與候選的實體向量空間模型計算相似度,取最相似的作為真正的結果。

% 這個有點複雜呀Orz...
% \cite{freebase-google} 則是Google則利用Freebase對實體消歧義。

%
%   Pattern and Relation
%
\section{樣式與實體間關係}

實體間的關係是可能隨著時間而產生變化的,
知道哪些關係會變化對於協助知識庫的更新是有幫助的,可以更注意那些會變動的關係。
\cite{relationsByTime} 提到有些實體間的關係是相對恆常的,
例如《1Q84》的作者是村上春樹是不會改變的事實;
而有些關係則是隨著時間而變動的,例如美國的總統是布希,只有在某一段時間內是正確關係。
此研究將實體間的關係分為是否為恆常(Constant)或是否唯一(Unique)。
恆常是指是否這組關係不會隨著時間改變,唯一是指抽換實體後關係是否仍然正確。
此研究人工挑選1,000組關係並利用時間、實體出現在特定時段內的頻率、文法等特徵對關係進行分類。

\cite{reverb} 建立了一套開放資訊擷取系統(Open Information Extraction System),
透過動詞表示的詞彙與句法限制,以<arg1, relation, arg2>的形式自動擷取出實體間的關係,
不限於人工標定的關係。

而Wikipedia中,每一篇文章,或稱條目,是描寫一個特定的實體,
\cite{wisenet} 利用Wikipedia文章中連至其他條目的連結作為實體,
以此擷取條目間,也就是實體間的關係。
由於同一種關係可能由不同的語句樣式(Patterns)來表達,
此研究應用了Wikipedia條目分類資訊、樣式的前後文來分類同義樣式。

\cite{patty2012}及\cite{patty}建立了一個名為PATTY的分類集(Taxonomy),
將用來表達關係的樣式進行分類,將同義樣式合併,並附上實體類別資訊,
將關係表達為「<type 1> \emph{Pattern} <type 2>」,Type 1和Type 2是實體的類別,
而Pattern則由單字(Words)、詞性標記(Part-Of-Speech Tags)、萬用字元(Wildcards)所組成,
例如「<person>'s [adj] voice * <song>」。   % FIXME - Example

PATTY自Wikipedia、New York Times抽取句子,
以史丹佛剖析器(Stanford Parser\footnote{http://nlp.stanford.edu/software/lex-parser.shtml})對句子建立剖析樹,
以YAGO、Freebase作為實體的字典,判斷若句子中有兩個實體,
則把兩實體間在樹上的最短路徑上的字句擷取出來作為樣式,
並進一步合併樣式為同義樣式集(Pattern Synset)。
比起\cite{reverb},PATTY可以抽取任意的關係,不被詞彙或句法所限制。

對每一個樣式,都存在一組支持集(Support Set),
由符合樣式的實體對(Pairs of entites)所組成。
透過支持集的大小,對每個樣式計算計算了一個信心值(Confidence),
將樣式中的實體類別從類別改為類別繼承架構中的更廣義的父節點類別,
以可以填入的實體對數作為分母,支持集作為分子,
得到的分數就是心信值,代表一個樣式的品質。

除了同義樣式集之外,此研究還做了關係釋義(Relation Paraphrasing):
給定一個來自知識庫的關係,判斷一個樣式是否可以描述此一關係。
PATTY釋義了225種DBpedia關係,包含127,811個樣式;25種YAGO關係,包含43,124個樣式。
其透過隨機選取1,000組釋義來評估,平均的精確度(Precision)是0.76$\pm$0.03。

本研究將嘗試利用PATTY提供的關係釋義來偵測實體特性是否存在於文件之中。   % FIXME - 收尾好像不是收的挺好

%\cite{aptagger}


%%
%   Chapter Method
%
%   Qing-Cheng Li (r01922024 at csie dot ntu dot edu dot tw)
%   R.O.C.103.07
%
\chapter{研究方法}
\label{c:method}

本研究欲自內容串流中偵測實體的特性,
而實體的特性之一:實體間關係,則可以語句樣式表現,
因此本研究主要利用語句樣式來偵測這樣的特性是否存在。
本章節將介紹所使用的資源、
樣式比對、樣式篩選與特性歧異的問題與解決方法。

\section{以樣式偵測特性}
當我們在字裡行間要描述一個實體的特性時,
應該會有某些特定的樣式。
例如我們知道Jobs的出生地這個特性是San Francisco,
我們可能會以「Jobs was born in San Francisco」這樣的句子來表達這個概念。
而在上述句子之中「was born in」就是一個可以用來表示出生地此一實體特性的樣式,
當我們在句子中看到「was born in」時,就可以推測這個句子存在某人的出生地是某地這樣的實體特性。

基於這樣的想法,本研究將利用樣式的出現有無來決定是否存在實體特性,
設計了如圖\ref{i:process-v1}的偵測流程。

圖\ref{i:process-v1}中,首先要先有一份實體特徵與樣式的關聯表,
知道每個實體特徵可以由哪些樣式來表達。
再利用這份關係表,對文章中的句子進行比對,檢查關係表中的樣式是否有出現,
若有出現,則這篇文章可能包含樣式對映的實體特性。
樣式比對的細節會在第\ref{s:pattern-match}節中詳述。

\begin{figure}
    \centering
    \includegraphics[width=0.9\textwidth]{images/03-process-v1}
    \caption{以樣式偵測特性流程概念}
    \label{i:process-v1}
\end{figure}

在這個流程之中,必須要有一份實體特徵與樣式的對映表。
此一部份本研究將利用PATTY已經提供的關係釋義表,
表中包含了知識庫YAGO內定義的實體間關係、
YAGO關係的領域與範圍(表\ref{t:yago-relation}),
以及可以用來描述這個關係的樣式。

%t:yago-relation
\begin{table}[t]
    \begin{center}
        \footnotesize
        \begin{tabular}{l||c|c|c}
        YAGO Relation & Domain & Range & Number of patterns \\
        \hline
        actedIn & wordnet\_actor & wordnet\_movie & 2023 \\
        created & yagoLegalActor & Thing & 3215 \\
        dealsWith & wordnet\_location & wordnet\_location & 366 \\
        diedIn & wordnet\_person & wordnet\_city & 1352 \\
        directed & wordnet\_person & wordnet\_movie & 1228 \\
        graduatedFrom & wordnet\_person & wordnet\_university & 2129 \\
        happenedIn & wordnet\_event & yagoGeoEntity & 47 \\
        hasAcademicAdvisor & wordnet\_person & wordnet\_person & 632 \\
        hasCapital & wordnet\_location & wordnet\_location & 24 \\
        hasChild & wordnet\_person & wordnet\_person & 3620 \\
        hasWonPrize & yagoLegalActorGeo & wordnet\_award & 78 \\
        holdsPoliticalPosition & wordnet\_person & wordnet\_person & 1173 \\
        influences & wordnet\_person & wordnet\_person & 2461 \\
        isCitizenOf & wordnet\_person & wordnet\_country & 813 \\
        isKnownFor & yagoLegalActor & Thing & 1574 \\
        isLeaderOf & wordnet\_person & yagoLegalActorGeo & 465 \\
        isLocatedIn & yagoPermanentlyLocatedEntity & yagoGeoEntity & 1300 \\
        isMarriedTo & wordnet\_person & wordnet\_person & 4276 \\
        isPoliticianOf & wordnet\_person & wiki\_states\_of\_US & 465 \\
        livesIn & wordnet\_person & wordnet\_location & 718 \\
        participatedIn & yagoLegalActorGeo & Thing & 89 \\
        playsFor & wordnet\_person & wordnet\_organization & 1491 \\
        wasBornIn & wordnet\_person & wordnet\_city & 1226 \\
        worksAt & wordnet\_person & wordnet\_organization & 1602 \\
        \end{tabular}
        \caption{YAGO Relations in PATTY}
        \label{t:yago-relation}
    \end{center}
\end{table}


光是這樣的流程並不足以偵測實體特性。
使用樣式來偵測實體特性還會有樣式覆蓋度(Coverage)、
品質(Quality)、可信賴度(Reliability)與歧異性(Ambiguity)的問題。

樣式覆蓋度問題是指對於某個實體的特性,
在關係表內屬於這個實體特性的樣式佔所有可以表達該特性的樣式之比例。
例如一個實體特性可以用100種樣式來描述,
但只找出了其中70種,
這樣就沒有辦法找到以其他30種樣式來描述該特性的句子。

樣式的品質則是一個字串到底能不能夠作為一個樣式,    %FIXME
一個樣式的品質不高,那麼這樣的樣式有可能太廣泛的出現在句子之中,
致使這樣的樣式沒有辦法區辨出實體屬性。
例如PATTY關係釋義中的DBpedia:manager內的樣式「hired」就只有這一個單字,
用以描述實體特性顯得太廣泛了些,
以此擷取回的文章有包含實體特性的比例就較低。

樣式可信賴度則是一個樣式到底能不能用來表達實體特性。
例如,對於特性YAGO:diedIn,以「since worked in」來描述完全文不對題。

而為了處理樣式的品質、可信賴度問題,
應該要在圖\ref{i:process-v1}流程中的關聯表與樣式比對中增加一個樣式篩選的步驟,
篩選出可信賴、有一定品質的樣式來進行樣式比對,
此部份將於第\ref{s:select-pattern}小節說明詳細內容。

經過樣式比對後,每篇文章可能存在數個樣式,
如果在關聯表中每個樣式只對映到一個關係,
那便沒有歧義。
但一個樣式可能存在於多個實體特性的關聯表內,
例如樣式「first met with」就被PATTY認為可以表示5種特性(hasAcademicAdvisor、isKnownFor、isMarriedTo、influences、hasChild)。
因此在樣式被比對出來之後,整個偵測流程中還需要一個消歧義的步驟。
這個部份將在第\ref{s:pattern-disambiguity}節中說明。

補上了樣式篩選與消歧義後的流程圖應修正為圖\ref{i:process-v2}。
樣式比對使用的樣式是經過篩選的,
比對出來後還要進行消歧義才完成偵測實體特性之流程。
由於偵測實體特性的流程對於每一分文件都是相同,且彼此互相獨立,
因此可以平行處理文件,如圖\ref{i:process-parallel}所示,
處理內容串流的大量文件。

\begin{figure}
    \centering
    \includegraphics[width=0.9\textwidth]{images/03-process-v2}
    \caption{以樣式偵測特性流程}
    \label{i:process-v2}
\end{figure}

\begin{figure}
    \centering
    \includegraphics[width=0.7\textwidth]{images/00-tmp-img}    %TODO
    \caption{平行處理架構}
    \label{i:process-parallel}
\end{figure}

\section{樣式比對}
\label{s:pattern-match}

樣式比對是要從句子之中確定某一個樣式是否存在,
PATTY中提供的25個YAGO關係有43,124個樣式,
但其中有不少是重複出現的,
實際上總共有19,031個獨特的樣式,
如果對每個句子都做19,031次比較顯然不是一個好方法。

% Prefix-Tree
為了讓樣式比對可以有效率些,
我們將所有的樣式建立成一顆前綴樹(Prefix Tree)。
先將所有首字相同的樣式合併成同一個節點,
再看第二個字是否相同,如果不同就分支出去,
以此類推,最後再連結到樣式編號,
即可透過編號查詢該樣式可能表示的實體特性。
如圖\ref{i:pattern-prefix-tree}所示,
四個樣式「was born」、「was born at」、「was born [[det]] village」和「was bron [[det]] willage in」的首字皆為was,所以併成一個節點。
第二個字也都相同,所以也併成一個節點。
而此時「was born」已經完成了,所以born節點下新增一個樣式編號的節點。
剩下三個樣式的第三個字有詞性標記「[[det]]」和單字「at」,
便在born節點下新增[[det]]和at節點。at節點完成了「was bron at」樣式,
而[[det]]節點則繼續將後續的字加入這棵前綴樹上,
最後這棵樹上所有葉子都是一個樣式編號。

\begin{figure}
    \centering
    \includegraphics[width=0.9\textwidth]{images/03-pattern-prefix-tree}
    \caption{樣式前綴樹}
    \label{i:pattern-prefix-tree}
\end{figure}

% Algo
有了樣式前綴樹後,
再來就是利用這棵樹來比對句子中是否有出現其中的樣式。
演算法\ref{a:pattern-match}描述了比對的過程。

\begin{algorithm}
    \caption{樣式比對演算法}
    \label{a:pattern-match}
    \begin{algorithmic}[1]
        \Require  
            S: 句子;
            T: 樣式前綴樹;
        \Ensure
            P: 有出現的樣式
        \State 初始化P=[], 暫存陣列tmpP=[]
        \State 對S進行詞性標記
        \For{i from 0 to length of S}
            \State (word,POStag) $\gets$ S[i]
        %\ForAll{(word,POStag) {\bf in} S}
            \ForAll{possiblePattern {\bf in} tmpP}
                \If{word or POStag {\bf in} T's possiblePattern.depthInTree+1 level nodes}
                    \State continue
                \Else
                    \If{T's possiblePattern.depthInTree+1 level node is PatternID}
                        \State {\bf add} (PatternID,startPoint,i as endPoint) {\bf into} P
                    \EndIf
                    \State {\bf remove} possiblePattern {\bf from} tmpP
                \EndIf
            \EndFor
            \If{word or POStag {\bf in} T's first level nodes}
            \State {\bf add} (depthInTree,i as startPoint) {\bf into} tmpP
            \EndIf
        \EndFor
        
        % 移除重疊的短樣式,只留重疊中最長的。TODO
        %\State {\bf sort} P by startPoint
        %\ForAll{pattern in P}
        %    \If{pattern overlap with other patterns}
        %
        %    
        %\EndFor

        \State \Return P
    \end{algorithmic}
\end{algorithm}
% Try to use Algorithm package?

% Complexity


\section{樣式篩選}
\label{s:select-pattern}

\section{特性消歧義}
\label{s:pattern-disambiguity}

% Pattern Extraction
% Pattern Matching
%   O(mn)?
% Some statistics e.g. how many pattern, how long, ...etc


%%
%   Chapter Experiment
%
%   Qing-Cheng Li (r01922024 at csie dot ntu dot edu dot tw)
%   R.O.C.103.07
%
\chapter{實驗結果與分析}
\label{c:exp}

本章節進行了以樣式偵測特性的實驗,
以Wikipedia的文章作為模擬內容串流文件、
以YAGO的資料作為答案、
以PATTY提供的樣式對Wikipedia的文章進行偵測。
並介紹評估的標準以及對實驗結果的分析。

\section{測試資料集}
\label{s:dataset}

為了模擬內容串流文件,
本研究以Wikipedia的條目文章作為內容串流中的文件,
自2013年3月的Wikipedia Dumps\footnote{http://dumps.wikimedia.org/}擷取文件。

由於一開始並無文件中包含哪些實體特性的正確資料,
而YAGO的YAGO Facts提供了<subject, property, object>的資訊,
因此我們使用YAGO Facts的propety作為參考答案,
並將subject連結至Wikipedia的文章,
這樣就有每篇文件具有哪些特性的參考答案。

樣式則是使用PATTY提供的樣式釋義,其包含了知識庫定義的關係,
即本研究擬偵測之實體特性,以及可用以表達該特性之樣式集。
實驗採用的是其中的YAGO關係(YAGO Relations),一共有25組關係,
但其中一組關係(YAGO:Produced)在YAGO中沒有找到對映的Wikipedia條目,
因此僅對餘下24組關係進行實驗。

接下來只留下存在這24種特性的Wikipedia條目,
為了防止YAGO Facts中存在某特性但文章中卻根本沒有提及的狀況發生,
我們只留下<subject,property,object>中subject條目內有出現object字詞片段的特性留下,
若object內的字根本沒有出現在文章中,則捨去這個特性。
經過處理後,最後共有334,469篇條目作為測試資料集。
表\ref{t:yago-coverage}統計了在不同的信心值下與不同的樣式歧義下,
樣式的數量以及含有這些樣式的文件總數。

%t:yago-coverage
\begin{table}[t]
    \begin{center}
        \small
        \begin{tabular}{l||c|c|c|c||c|c|c|c}
          & \multicolumn{4}{ c|| }{信心值> 0} & \multicolumn{4}{ c }{信心值> 0.7} \\
        \hline
        歧義度 & 樣式數 & 出現 & 涵蓋文章 & 比例
            & 樣式數 & 出現 & 涵蓋文章 & 比例 \\
        \hline
        1   & 11381 & 7267  & 250737    & 74.97 & 8913  & 5854  & 244888    & 73.22 \\
        2   & 4778  & 2976  & 275736    & 82.44 & 4175  & 2697  & 272395    & 81.44 \\
        3   & 1255  & 886   & 278820    & 83.36 & 1048  & 745   & 274265    & 82.00 \\
        4   & 666   & 503   & 281256    & 84.09 & 600   & 458   & 276782    & 82.75 \\
        5   & 635   & 465   & 283830    & 84.86 & 605   & 442   & 279603    & 83.60 \\
        6   & 77    & 64    & 283858    & 84.87 & 68    & 57    & 279648    & 83.61 \\
        7   & 132   & 100   & 284128    & 84.95 & 131   & 100   & 280096    & 83.74 \\
        8   & 49    & 43    & 284275    & 84.99 & 49    & 43    & 280342    & 83.82 \\
        9   & 26    & 25    & 284295    & 85.00 & 26    & 25    & 280385    & 83.83 \\
        10  & 13    & 11    & 284297    & 85.00 & 13    & 11    & 280387    & 83.83 \\
        11  & 12    & 9 & 284299    & 85.00 & 12    & 9 & 280391    & 83.83 \\
        12  & 2 & 2 & 284299    & 85.00 & 2 & 2 & 280391    & 83.83 \\
        13  & 0 & 0 & 284299    & 85.00 & 0 & 0 & 280391    & 83.83 \\
        14  & 2 & 2 & 284299    & 85.00 & 2 & 2 & 280391    & 83.83 \\
        15  & 0 & 0 & 284299    & 85.00 & 0 & 0 & 280391    & 83.83 \\
        16  & 0 & 0 & 284299    & 85.00 & 0 & 0 & 280391    & 83.83 \\
        17  & 3 & 3 & 284299    & 85.00 & 3 & 3 & 280391    & 83.83 \\
        \hline
        總計 & 19031 & 12356    &  &    &15647 &10448 & & \\
        \hline
        \hline
          & \multicolumn{4}{ c|| }{信心值> 0.8} & \multicolumn{4}{ c }{信心值> 0.9} \\
        \hline
        歧義度 & 樣式數 & 出現 & 涵蓋文章 & 比例
            & 樣式數 & 出現 & 涵蓋文章 & 比例 \\
        \hline
        1   & 5951  & 4029  & 222371    & 66.48 & 1879  & 1121  & 155402    & 46.46 \\
        2   & 2328  & 1697  & 251552    & 75.21 & 316   & 206   & 163265    & 48.81 \\
        3   & 683   & 487   & 258715    & 77.35 & 187   & 110   & 170820    & 51.07 \\
        4   & 473   & 355   & 261633    & 78.22 & 67    & 47    & 171978    & 51.42 \\
        5   & 278   & 243   & 262321    & 78.43 & 18    & 16    & 173397    & 51.84 \\
        6   & 50    & 43    & 262364    & 78.44 & 10    & 8 & 173457    & 51.86 \\
        7   & 52    & 41    & 262452    & 78.47 & 12    & 8 & 174396    & 52.14 \\
        8   & 13    & 15    & 263419    & 78.76 & 2 & 2 & 174476    & 52.17 \\
        9   & 12    & 12    & 263514    & 78.79 & 8 & 8 & 175287    & 52.41 \\
        \hline
        總計    & 9840  & 6922  &   &   & 2499  & 1526  &   & \\
        \end{tabular}
        \caption{樣式總數量、出現數量、涵蓋文章與比例於不同信心值之統計}
        \label{t:yago-coverage}
    \end{center}
\end{table}


由於資料集內的Wikipeida條目皆是描寫實體,
在假設文章的內容都是以描寫該實體的前提下,
可以利用YAGO Simple Types作為該文描述實體之類型,
搭配樣式的領域(Domain)資訊輔助偵測。

\section{評估標準}
\label{s:eval}
在偵測系統的效能方面,
我們希望了解對每一個實體特性,
利用樣式自文章中偵測該特性的效能。
因此,以精確率(Precision)、召回率(Recall)、$F_1$分數($F_1$ Score)進行評估。
精確率公式如式\ref{f:precision},召回率公式如式\ref{f:recall}。

\begin{equation}
    \label{f:precision}
    Precision = \frac{|\{relevant\ documents\}\cap\{retrived\ documents\}|}{|\{retrived\ documents\}|}
\end{equation}

\begin{equation}
    \label{f:recall}
    Recall = \frac{|\{relevant\ documents\}\cap\{retrived\ documents\}|}{|\{relevant\ documents\}|}
\end{equation}

對於一個實體特性,經過偵測之後會有被標為有此特性的文件與無此特性的文件。
其中,相關的文件(Relevant documents)即真正存在該特性之文件;
尋回的文件(Retrived documents)即偵測系統標記為擁有此特性之文件。
精確率評估在尋回的文件之中有多少文件真正存在該特性;
召回率評估在真正擁有該特性的文件中有多少被系統偵測到。

$F_1$分數則是綜合評估精確率與召回率,計算方式如式\ref{f:f1},為精確率與召回率的調和平均。

\begin{equation}
    \label{f:f1}
    F_1\ Score = \frac{2\times Precision \times Recall}{Precision + Recall}
\end{equation}

除了評估個別特性的效能之外,以宏觀平均(Macro Average)及微觀平均(Mirco Average)分別計算精確率與召回率及$F_1$分數來評估整體效能。
以精確率為例,宏觀平均的計算如式\ref{f:macro},
而微觀平均的計算則如式\ref{f:micro}。

\begin{equation}
    \label{f:macro}
    Macro\ Avg\ Precision=\frac{\sum_i^n precision_i}{n}
\end{equation}

\begin{equation}
    \label{f:micro}
    Mirco\ Avg\ Precision=\frac{\sum_i^n |\{relevant\ documents\}_i|\cap|\{retrived\ documents\}_i|}{\sum_i^n |\{retrived\ documents\}_i|}
\end{equation}


\section{實驗結果}
\label{s:result}

%\subsecion{}
%錯誤分析

%%
%   Chapter Conclusion
%
%   Qing-Cheng Li (r01922024 at csie dot ntu dot edu dot tw)
%   R.O.C.103.07
%
\chapter{結論與未來展望}
\label{c:future}

\section{結論}
有別於過去的研究多半著重於於內容串流中找尋實體相關的文件,
本研究專注於於內容串流中尋找實體的特性,
提出了於基於樣式於內容串流中偵測實體特性的方式。
自內容串流中比對樣式,透過實體特性與樣式間的關聯,
快速地偵測文件是否包含特定的實體樣式。
由於偵測流程文件與文件是互相獨立的,
整個偵測的流程可平行化,具備可擴展性,
能夠處理模擬真實世界的內容串流。

本研究也為偵測特性的過程加入了樣式篩選與特性消歧義等步驟以提升偵測效能。
樣式篩選針對了樣式信心值、可信賴程度以及樣式歧義度三個面向進行篩選。
對樣式篩選越嚴格,則可以使用的樣式越少,因此召回率會下降。
例如對信心值進行篩選,精確率有提升,但整體效能仍下降;
對可信賴度進行篩選,精確率隨有提升但隨著門檻提高,太高的門檻反而使整體效能下滑。
而採用歧義度越高的樣式,則可採用的樣式總數增加,召回率有提升,而整體效能緩步下滑。

特性消歧義的步驟則是進行於樣式比對之後,
用以區辨樣式出現的地方是否無實體特性或有實體特性,是一個或多個實體特性。
透過引入實體類型資訊捨去不符合條件的特性,可在不改變召回率的情形下使系統的效能有顯著的提升。
針對特性的篩選實驗了四種篩選策略,由宏觀平均的角度來看,選擇正規化後正確率大於0.5的特性是最有幫助的。
最後再透過簡單貝氏分類器對特性進行存在或不存在的二元分類,可以顯著提升系統效能。
總體而論,結合引入實體類型資訊、選擇正規化後正確率大於0.5的特性後以簡單貝氏分類器進行分類可以得到最好的系統效能。

\section{未來展望}
就偵測流程本身,分析錯誤主要來自三類。第一類是來自歧義性,
第二類是無法分辨有無特性存在,以及第三類覆蓋度不足。
其中以第二類為影響精確率的主要錯誤來源,
如何消彌這幾種錯誤以繼續改善偵測效能是未來的研究課題之一。

而本研究所使用之樣式,以及樣式與實體特性之關聯表來自於PATTY,
在樣式的覆蓋度以及與實體特性間的關聯度強弱受到一定程度的限制,
在未來希望可以自網路資源擷取樣式降低覆蓋度的影響,
並與知識庫特性建立更細緻的關聯,以降低歧義性造成的影響。

TREC知識庫加速競賽專注於串流文件中尋找目標實體相關的文件,
並判斷是否有助於更新知識庫,而本研究專注於文件中尋找實體特性,
若本研究能搭配知識庫加速競賽之實體尋找,
又能知道何種實體特性是會隨時間而有所變化,
就能更容易的判斷是否有新知識的產生或變動。

關於測試資料集,因為沒有正確的答案標記,只能透過YAGO來進行推測,
並假設所有的語句都是描寫該條目,希望未來可以針對文件,
甚至到句子等級的答案標記,可以進行更詳細的研究。
本研究目前是針對文件等級進行偵測,若有更細緻的答案標記,
或許可以進行句子等級的偵測。
更進一步,希望能夠知道句子中樣式描述的對象實體與其在句子中的位置,
才能更精確地偵測實體特性。
而Wikipedia的條目屬於較長的文章,也希望未來可以針對微網誌、社群留言等短訊息進行偵測。
使得偵測的面向更為廣泛以及更為細緻,甚至是自動化地對知識庫進行更新。



\appendix

\backmatter

\addcontentsline{toc}{chapter}{\bibname}
\bibliographystyle{apa}

% Your bibliography goes here
\bibliography{thesis}

\end{document}
